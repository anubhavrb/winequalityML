
% The \section{} command formats and sets the title of this
% section. We'll deal with labels later.
\section{Introduction}
\label{sec:intro}

In this section, you should introduce the reader to the problem you
are attempting to solve. For example, for the first project: describe
the dataset, and the prediction problem that you are
investigating. You should also cite and briefly describe other related
papers that have tackled this problem (or similar ones) in the past
--- things that came up during the course of your research. In the
AAAI style, citations look like \cite{aima} (see
the comments in the source file \texttt{intro.tex} to see how this
citation was produced). Conclude by summarizing how the
remainder of the paper is organized. \\

% Citations: As you can see above, you create a citation by using the
% \cite{} command. Inside the braces, you provide a "key" that is
% uniue to the paper/book/resource you are citing. How do you
% associate a key with a specific paper? You do so in a separate bib
% file --- for this document, the bib file is called
% project1.bib. Open that file to continue reading...

% Note that merely hitting the "return" key will not start a new line
% in LaTeX. To break a line, you need to end it with \\. To begin a 
% new paragraph, end a line with \\, leave a blank
% line, and then start the next line (like in this example).
Overall, the aim in this section is context-setting: what is the
big-picture surrounding the problem you are tackling here?

